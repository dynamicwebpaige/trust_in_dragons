
% This is a LaTex documentation file for the
% kjbuckets C extension to Python by Aaron Watters.
%
% The kjbuckets package consists of the following files
%   COPYRIGHT - a copyright notice.
%   kjbuckets.README - a readme.
%   kjbuckets.tex - documentation (this file).
%   kjbucketsmodule.c - the C source for the module
%   kjtest.py - some test results comparing access speeds
%        for kjTables under various settings for GSIZE.
%
\documentstyle{article}
\textwidth=6.5in
\textheight=8.5in
\topmargin=0.0in
\oddsidemargin=0in
\evensidemargin=0in
\markright{Watters -- kjbuckets 2.2 -- draft}
\pagestyle{myheadings}
\title{Set and Graph Datatypes for Python\\
       Release 2.2}
\author{Aaron Watters\\
	Computer and Information Sciences\\
	New Jersey Institute of Technology\\
	University Heights\\
	Newark, NJ, 07102\\
	(201)596-2666\\
	watters@vienna.njit.edu
	}
\begin{document}           % End of preamble and beginning of text.
\maketitle

\begin{abstract}
This is the documentation for the {\tt kjbuckets} C extension to
Python (second release), 
which defines graph and set datatypes as well as an
alternative dictionary data type.  These types are tightly coupled
at the level of C, allowing fast and powerful algebraic combinations
of container objects.
\end{abstract}

\section{Introduction}

The {\tt kjbuckets} module defines three data types for Python: {\tt
kjSet}, {\tt kjGraph}, and {\tt kjDict}.  These types come with a
number of associated methods, including common set theoretical
operations such as union, intersection, difference, composition,
transposition, reachability sets, and transitive closure.

For suitably large compute intensive uses these types should provide
up to an order of magnitude speedup versus an implementation that uses
analogous operations implemented directly in Python.

The following discussion assumes the {\tt kjbuckets} module has been
compiled and installed in the Python executable.  For information on
how to perform such an installation, see the Python extensions manual
that comes with the Python distribution.

Release 2.2 contains a number of goodies not documented here.
If you want, you can try to figure them out from looking at the
code!

Release 2.1 had a problem linking under Python 1.2.
This has been fixed in 2.2.

\section{The Types}

This module defines three types
\begin{description}
\item[kjSets]
are initialized using the function {\tt kjbuckets.kjSet()}.  They are
containers for Python hashable objects with no significance to
redundancy and no order to members. EG:\footnote{Most of the
examples given here use numeric elements for ease of presentation, 
which is bad because it's
boring.  It's also bad because it leaves the impression that only
simple things can be archived -- which is wrong.  Remember that
keys may be any hashable type (which even includes user defined
classes which have a hash method defined), and for dictionaries
and graphs the left members may be any Python object whatsoever.}
\begin{verbatim}
  >>> from kjbuckets import *
  >>> X = kjSet([1,2,3,3,5,4]); print X
  kjSet([1, 4, 3, 2, 5])
  >>> Y = kjSet([5,5,3,3,2,1,4,4,4]); print Y
  kjSet([1, 4, 3, 5, 2])
  >>> X == Y
  1
\end{verbatim}
\item[kjGraphs]
are initialized using the function {\tt kjbuckets.kjGraph()}.  They
relate Python hashable objects to other objects, with no
significance to order or redundancies on the pairings.  
Technically, kjGraph defines a directed graph abstract data type.
EG:
\begin{verbatim}
  >>> G1 = kjGraph([(1,1),(1,2),(2,4),(9,6),(2,4)]); print G1
  kjGraph([(1, 1), (1, 2), (9, 6), (2, 4)])
  >>> G1.reachable(1)
  kjSet([1, 4, 2])
\end{verbatim}
\item[kjDicts]
are initialized using the function {\tt kjbuckets.kjDict()}.
They map hashable objects to other objects, in a manner
similar to the Python builtin Dictionary data type,
except that the {\tt kjbucket} implementation is 
slower.  That is, it is slower if you use it
just like another Python dictionary.  It's a lot faster
if you want to do compositions, intersections, and so forth
using dictionaries.\footnote{And with the new release the speed
difference is not so great anymore -- about 20\% slower on
comparable operations -- and kjDict's tend to use
less space than Python dictionaries for the same contents.}
\begin{verbatim}
  >>> D = kjDict([(1,1),(1,2),(2,4),(9,6),(2,4)]); print D
  kjDict([(1, 2), (9, 6), (2, 4)])
  >>> D * D
  kjDict([(1, 4)])
\end{verbatim}
\end{description}

\section{Initialization functions}

Each of the initialization functions accept four possible argument
sequences:
\begin{description}
\item[No argument:]
Results in the creation of a smallest empty object of the requested
type.
For example {\tt kjSet()}, creates the smallest possible empty kjSet.
\item[Contents list or tuple:]
As illustrated above, the structures may be initialized with
a list or tuple of contents, where the elements of the sequence are
tuples of form
(hashable object, object) pairs for kjDicts and kjGraphs and
just hashable objects for kjSets.  The examples given here use lists
as the top level structure for the sequence initialization form,
but you can also use tuples.  For example as in
\begin{verbatim}
   >>> kjDict( ( (1,2), (2,3), (2,4), (3,4) ) )
   kjDict([(1, 2), (2, 4), (3, 4)])
   >>> kjSet( (9,2,1,9,8,7,6,4) )
   kjSet([9, 6, 1, 7, 4, 2, 8])
\end{verbatim}
In the case of kjDicts if there are key collisions the
resulting kjDict may be dirty.
\item[Other kjTable:]
If the initializer argument is another kjTable the result will be the input
table ``coerced'' to the other type (or if the types match
you will get ``first-level'' copy of the table.  
The new object will be a distinct table which shares object
references with the input table.
\begin{verbatim}
   >>> G
   kjGraph([(0, 0), (1, 1), (2, 2), (3, 3), (4, 4), (0, 5), 
            (1, 0), (2, 1)])
   >>> kjDict(G)
   kjDict([(0, 5), (1, 0), (2, 1), (3, 3), (4, 4)])
   >>> kjSet(G)
   kjSet([0, 1, 2, 3, 4])
   >>> G2=kjGraph(G)
   >>> G2
   kjGraph([(0, 0), (1, 1), (2, 2), (3, 3), (4, 4), (0, 5), 
           (1, 0), (2, 1)])
   >>> G[12]=3
   >>> G
   kjGraph([(0, 0), (1, 1), (2, 2), (3, 3), (4, 4), (0, 5), 
            (1, 0), (2, 1), (12, 3)])
   >>> G2
   kjGraph([(0, 0), (1, 1), (2, 2), (3, 3), (4, 4), (0, 5), 
            (1, 0), (2, 1)])
\end{verbatim}
Coercing
a graph to a dictionary where the graph maps the same object to several
objects will produce
a ``dirty'' dictionary with key collisions decided arbitrarily.
Coercing a set to a graph or dictionary produces an ``identity''
containing {\tt (x,x)} for each element {\tt x} of the Set.
Coercing a graph or dictionary to a set produces the set of
keys (left members) from the graph or dictionary.  To get the
``set of arcs'' from a graph use {\tt kjSet(G.items())} instead
of {\tt kjSet(G)}.
\item[Number:]
This option refers to the internal implementation of the types.
Internally these types are implemented using arrays.  Sometimes
these arrays need to be resized to a larger size before an
insert can complete.
By initializing using a single integer argument $n$, you request
that the structure be large enough that no resize will be needed
until after $n$ inserts (in the absense of deletions).  
For example {\tt S = kjSet(1000)} initializes
a set that will not need to be resized until after 1000 inserts
have completed.\footnote{However, since deletes sometimes trigger
the array to resize to a smaller size, deleting an element from
{\tt S} before insert number 1000 may make resizing necessary anyway.
Them's the breaks.}  Using this option may save some time and
prevent some unnecessary memory fragmentation, when the programmer
can determine (or guess) the expected number of insertions, a priori.
\end{description}

There is a peculiar way to initialize a {\tt kjDict}:
\begin{verbatim}
   >>> kjUndump(("name","age"), ("aaron",12))
   kjDict([('name', 'aaron'), ('age', 12)])
   >>> kjUndump(("ssnum",),"123456789")
   kjDict([('ssnum', '123456789')])
\end{verbatim}
This is a parallel operation to {\tt kjDict.dump} which together
are designed to make it easy to pack and unpack information from
{\tt kjDict}s, in particular for constructing database-style
indices.  There are two behaviors for this function.
Called with arguments of form 
\begin{verbatim}
   kjUndump( (key,), map )
\end{verbatim}
(ie, the first argument is a tuple of length one and map is any object)
the result is the same as
\begin{verbatim}
   kjDict( [ (key, map) ] )
\end{verbatim}
Alternatively, called with two tuples of the same length with
lengths larger than 1 the invocation
\begin{verbatim}
   kjUndump( (k1, k2, ..., kn), (m1, m2, ..., mn) )
\end{verbatim}
produces the same result as
\begin{verbatim}
   kjDict( [ (k1,m1), (k2,m2), ..., (kn, mn) ] )
\end{verbatim}
If the same key is mentioned twice in the first argument
and the corresponding values in the second argument are
not equal the result will be a dirty dictionary.

\section{Dirtiness}

A table which has had a non-monotone update (ie, a deletion
or a dictionary overwrite) is said to be ``dirty.''  In particular
any deletion makes a table dirty; and coercing a graph to a dictionary,
or transposing a dictionary, or unioning a set or dictionary with
a dictionary will produce dirty dictionaries if the
computation results in any key
collisions.  To test whether a table is dirty use the {\tt X.Clean()}
method which produces {\tt X} if {\tt X} is clean, otherwise None.
For example:
\begin{verbatim}
   >>> G = kjGraph([(0, 0), (0, 1), (1, 4), (9, 9), (2, 5)])
   >>> D = kjDict(G); print D; print D.Clean()
   kjDict([(0, 1), (1, 4), (9, 9), (2, 5)])
   None
   >>> D2 = kjDict(D); print D2.Clean()
   kjDict([(0, 1), (1, 4), (9, 9), (2, 5)])
\end{verbatim}
Here {\tt D} is dirty because the coercion from a graph resulted in
key collisions on {\tt 0}, but the fresh copy {\tt D2} is not dirty.
The result of an algebraic expression involving a dirty
table will be dirty also, for example
\begin{verbatim}
  >>> D3 = D2 * D
  >>> print D3, D3.Clean()
  kjDict([(0, 4), (9, 9)]) None
\end{verbatim}
Note that, for example {\tt kjDict([(1,2),(1,3)])} will be dirty
but {\tt kjDict([(1,2),(1,2)])} is not, i.e., inserting the
same pair twice is not considered a collision.

These types have a number of associated methods, operations, and accessors.
For the purposes of discussion assume that {\tt S} is a kjSet,
{\tt D} is a kjDict, and {\tt G} is a kjGraph in the remainder.  
Furthermore assume {\tt X} is an object of
any of these types.

\section{Methods}

There are a number of methods associated with each member of these
types.
\begin{description}
%
\item{\tt S.member(ob), D.member(arg,map), G.member(src,dst)}
respectively are membership tests for the types.  Each returns 1 if the object
or pair are members of the structure or 0 otherwise.
%
\item{\tt S.add(ob), D.add(arg,map), G.add(src,dst)}
respectively add new members to the object. These are equivalent to
{\tt G[src]=dst, D[arg]=map, S[ob]=1} but the former may be
preferrable for graphs and sets since they are less misleading.
This is an ``in place'' mutation operation -- it will raise an
error if the object has been hashed.
%
\item{\tt D.delete\_arc(arg,map), G.delete\_arc(src,dst)}
respectively delete a pair from the structure or raise an
error if the pair is not found.
This is an ``in place'' mutation operation -- it will raise an
error if the object has been hashed.
%
\item{\tt X.has\_key(key)}
determines whether a given key value occurs in the structure.
In the case of sets this is identical to the membership test.
In the case of dictionaries and graphs the function tests
whether {\tt key} occurs as a left member of some pair in the
structure and returns 1 if so, otherwise 0.
%
\item{\tt X.choose\_key()} selects an arbitrary key from the structure.
In the case of sets it returns an arbitrary member of the set.
In the case of graphs and dictionaries it picks an arbitrary left
member of a pair in the structure.  This operation is useful for
algorithms that begin ``pick an arbitrary node of the graph...''
This method is ``nondeterministic'' in the sense that tables with
the same members may choose different keys.
%
\item{\tt X.subset(Y)}
determines whether {\tt X} is a subset of {\tt Y}.
Returns 1 if so, else 0.  {\tt X} and {\tt Y} may be of different
types but may be confusing if one argument is a set and the other
is not.
If {\tt X} is a set and {\tt Y} is a graph or dictionary
then {\tt subset} will succeed if and only if {\tt Y} contains {\tt (e,e)}
for each member {\tt e} of {\tt X}.
If {\tt Y} is a set and {\tt X} is a graph or dictionary
then {\tt subset} will succeed if and only if every key of
{\tt X} is a member of {\tt Y}.
%
\item{\tt G.neighbors(key)}
returns a list of the objects $y$ where {\tt (key, $y$)} is
a member of {\tt G.}  For example
\begin{verbatim}
  >>> G = kjGraph([(0, 0), (1, 1), (0, 4), (1, 5), (2, 2), (2, 6)])
  >>> G.neighbors(1)
  [1, 5]
\end{verbatim}
If the key is absent from the table the result will be the empty
list.
This method is also defined for dictionaries, where the only possible
results are a unary list if the key is present or an empty list if
the key is absent.
%
\item{\tt G.reachable(key)}
returns a kjSet of objects reachable on any path in the graph
that begins at {\tt key.}  The {\tt key} itself will occur in the
result only if it lies on a loop of the graph.  For example
\begin{verbatim}
>>> G = kjGraph([(1, 0), (4, 1), (0, 2), (3, 2), (6, 3), (2, 4), (5, 0)])
>>> G.reachable(5)
kjSet([0, 4, 1, 2])
\end{verbatim}
Again this method is also defined for dictionaries.  The method
returns a kjSet rather than a list because this made sense to me at
the time.
%
\item{\tt X.items()}
returns a list of the members of the structure.  For example:
\begin{verbatim}
  >>> X = kjSet([0, 1, 2, 0, 1])
  >>> X.items()
  [1, 0, 2]
  >>> X = kjGraph([(3, 0), (2, 2), (1, 2), (2, 0), (2, 0), (3, 0)])
  >>> X.items()
  [(1, 2), (3, 0), (2, 2), (2, 0)]
\end{verbatim}
\item{\tt G.keys(), G.values()}
return the left members and right members of pairs in the graph 
{\tt G} respectively.  For example:
\begin{verbatim}
   >>> G = kjGraph([(4, 8), (0, 9), (1, 10), (4, 9), (3, 7), (3, 8), (2, 7)])
   >>> G.keys()
   [4, 0, 1, 3, 2]
   >>> G.values()
   [8, 9, 10, 9, 7, 8, 7]
\end{verbatim}
Note that {\tt keys} eliminates redundancies, whereas {\tt values}
does not.
These functions are also defined for
dictionaries but are not defined for sets.  
%
\item{\tt S.ident()}
generates an ``identity dictionary'' from the set {\tt S}, the graph
containing exactly those members {\tt ($x$,$x$)} where $x$ is 
a member of {\tt S}.  For example, the following calculation
determines the ``self-loop'' elements of G:
\begin{verbatim}
   >>> G
   kjGraph([(0, 0), (0, 3), (0, 2), (1, 4), (9, 9), (2, 5)])
   >>> I = kjSet(G).ident()
   >>> I & G
   kjGraph([(0, 0), (9, 9)])
\end{verbatim}
(In the previous release {\tt ident} produced a graph,
but now that the algebraic operators have been generalized I
opted to produce the more specific dictionary type.  This operation
is now redundant since it is the same as {\tt kjDict(S)}.)
%
\item{\tt G.tclosure()}
generates the transitive closure graph 
derived from the graph {\tt G.}  
For example:
\begin{verbatim}
  >>> G = kjGraph([(1, 3), (4, 1), (3, 0), (3, 1)])
  >>> G.tclosure()
  kjGraph([(1, 3), (4, 1), (1, 0), (1, 1), (4, 3), 
           (3, 0), (3, 1), (3, 3), (4, 0)]
\end{verbatim}
%
\item{\tt X.Clean()}
produces None if table {\tt X} has experienced a non-monotone
update (a deletion or a dictionary key collision) or was
algebraically derived from a table that had experienced a
non-monotone update, in all other cases it returns the table
{\tt X} itself.  This is particularly useful for testing whether
the unions of dictionaries or the transpose of a dictionary
was unambiguous.
\begin{verbatim}
   >>> D = kjDict([('name', 'A. Watters'), ('ssn', 123)])
   >>> D2 = kjDict([('ssn', 999), ('salary', 9000000)])
   >>> D3 = D + D2; print D3
   kjDict([('name', 'A. Watters'), ('ssn', 999), ('salary', 9000000)])
   if D3.Clean() != None:
   ...    print D3["name"], " makes ", D3["salary"]
   ... else:
   ...    print "ambiguous dictionary union"
   ...
   ambiguous dictionary union
\end{verbatim}
Relational natural join anyone?
%
\item{\tt X.Wash(), X.Soil()}
force a table to appear to be clean or dirty respectively, both
returning {\tt None}.  Included for completeness.
%
\item{\tt D.remap(X)}
produces a dictionary that is the result of remapping
D by X, but it produces None if the remapping causes
a key collision.  For example to rename keys {\tt l} and
{\tt f} to {\tt lname} and {\tt fname} respectively, preserving
{\tt ssn}, equating {\tt ssn} with {\tt enum}, and disregarding
all other keys for D we could write.
\begin{verbatim}
   >>> D = kjDict( [("f","aaron"), ("l","watters"), ("m","robert"), 
                    ("ssn",123)] )
   >>> G = kjGraph()
   >>> G["ssn"]="enum"
   >>> G = (G + ~G).tclosure() # symmetric and transitive closure
   >>> G["lname"] = "l"; G["fname"] = "f"
   >>> D.remap(G)
   kjDict([('enum', 123), ('ssn', 123), ('lname', 'watters'), 
           ('fname', 'aaron')])
\end{verbatim}
This may seem strange, but it can be a very useful way
of transforming collections of dictionaries.
This operation is exactly the same as
{\tt kjDict(X*D).Clean()} but faster.  (I use
it a lot, so I optimized it -- it can correspond
to projection, equality selection, and renaming in
the relational algebra).  
%
\item{\tt D.dump(X)} 
packs right members of a dictionary into a compact form.
This function has two behaviors:
\begin{verbatim}
   >>> D = kjUndump(("name","age","ssn"), ("aaron",12,12345))
   >>> D
   kjDict([('name', 'aaron'), ('age', 12), ('ssn', 12345)])
   >>> D.dump(("ssn",))
   12345
   >>> D.dump(("name","ssn"))
   ('aaron', 12345)
\end{verbatim}
Called with an argument of form
\begin{verbatim}
    D.dump( (key,) )
\end{verbatim}
(ie, a tuple of length one) it produces the same result as
\begin{verbatim}
    D[key]
\end{verbatim}
Alternatively, called with an argument of form
\begin{verbatim}
    D.undump( (k1, k2, ..., kn) )
\end{verbatim}
(ie, a tuple of length greater than one) it produces that same
result as
\begin{verbatim}
    ( D[k1], D[k2], ..., D[kn] )
\end{verbatim}
This function is the parallel operation to
the dictionary initializer {\tt kjUndump}, which together
are designed to make it easy to pack and unpack information
from {\tt kjDict}s.  It is also defined on graphs, in which
case the choice of for the resulting mapped items may be
arbitrary.
%
\item{\tt len(X)}
return the number of entries in X (which is the
number of pairs in the case of
graphs or dictionaries).
%
\item{\tt del X[key]}
deletes the key from the structure.  In the case of sets, this
simply removes an element.  In the case of dictionaries and graphs
this method removes {\em all} entries with left member {\tt key.}
For example:
\begin{verbatim}
  >>> G = kjGraph([(1, 3), (4, 1), (3, 0), (3, 1)])
  >>> del G[3]
  >>> G
  kjGraph([(1, 3), (4, 1)])
\end{verbatim}
This is an ``in place'' mutation operation -- it will raise an
error if the object has been hashed.
\end{description}

\section{Hashing}

These types are hashable, that is, they may
be used as keys in hash structures and you may apply the
function {\tt hash(X)} to them.  The {\tt kjGraph}
and {\tt kjDict} structures also allow
hashing even if some of their right members are unhashable.
The ``down side'' of this ``hashing unhashables'' feature is
that if two structures of the same type only differ on their
unhashable right members they will hash to the same value --
which can make hash table look-ups slow.  A ``rule of thumb''
is to only use {\tt kjDict}s and {\tt kjGraph}s as keys of
a hash table structure if the set of keys is expected to 
nearly always differ
on hashable components.

{\bf However, once a table's
hash value has been computed for any reason, that table becomes
immutable -- any attempts to mutate the structure in place
(using index assignment, {\tt del}, 
{\tt X.delete\_arc}, or {\tt X.add})
will raise a {\tt TypeError}.}

\section{Other Properties}

Objects of these types may be compared
for equality where {\tt X==Y} succeeds if and only if X and Y
contain the same members.  Mixed type equality comparisons 
between kj-tables are
allowed, where if {\tt S==D} succeeds if and only if {\tt D}
consists of the pairs {\tt (e,e)} for each element {\tt e}
of {\tt S}, and similarly for {\tt S==G}.

Objects of these types may also be
used as booleans where only an empty structure is equivalent to
false.

One questionable aspect of the implementation is the use of 
the indexing notation.  Although it may be completely avoided,
both kjSets and kjGraphs allow indexing.  In the case of sets
{\tt S[object]=anything} inserts the {\tt object} as a member of the
set and disregards {\tt anything}, and a retrieval {\tt S[object]}
returns 1 if {\tt object} is a member of the set or raises an
key error otherwise.  For example,
\begin{verbatim}
  >>> S
  kjSet([1, 3, 2])
  >>> S["this"] = "that"
  >>> S
  kjSet([1, 3, 2, 'this'])
  >>> S["this"]
  1
  >>> S["that"]
  KeyError: that
\end{verbatim}
In the case of graphs {\tt G[object]=map} adds {\tt (object, map)}
as a new arc of the graph, and {\tt G[object]} retrieves an
arbitrary neighbor associated with  {\tt object}, or raises
a KeyError if there is none.  For example:
\begin{verbatim}
  >>> G
  kjGraph([(1, 3), (4, 1)])
  >>> G[1] = 9
  >>> G
  kjGraph([(1, 3), (4, 1), (1, 9)])
  >>> G[1]
  3
  >>> G[6]
  KeyError: 6
\end{verbatim}
Some may find this use of indexing notation non-intuitive, but
others may find it appealing, as far as I know.

Index assignment is an ``in place'' mutation operation -- it will raise an
error if the object has been hashed.

\section{Algebraic Operations}

The implementation provides a number of common set theoretical
operations over these structures.  All the set algebraic operations
are side effect free (and they may be applied to tables which have
been hashed).  These operations may be applied to tables with
differing types, except where noted.  
Except for
intersection and difference, a binary operation applied to objects
of different types produces an object of the ``more general'' type,
i.e, 
{\tt S+D} produces a (possibly dirty) dictionary,
{\tt S+G} produces a graph,
{\tt D+G} produces a graph.
Binary operations applied to objects of the same type produces
an object of that type.

Generally, when a set {\tt S}
is used in permitted mixed-mode algebra with a graph or a dictionary it
``acts like'' the identity dictionary {\tt S.ident()}.  

The built in algebraic operations are as follows.
\begin{description}
\item{Union}
produces the union of two structures of the same type, invoked
using either the notation {\tt X+Y} or {\tt X|Y}.  For example:
\begin{verbatim}
   >>> kjGraph([(1,3), (4,1), (1,9)]) + kjSet([6,7,2])
   kjGraph([(1, 3), (4, 1), (1, 9), (6, 6), (7, 7), (2, 2)])
\end{verbatim}
If dictionary {\tt D1} contains {\tt (key, map1)} and
dictionary (or set) {\tt D2} contains {\tt (key, map2)} then
{\tt D1+D2} will be a dirty dictionary containing
one of the pairs, but not the
other.
\item{Difference}
produces the set difference of two structures of the same type,
invoked using the notation {\tt X-Y}.  For example:
\begin{verbatim}
  >>> kjSet([1,2,5,7]) - kjSet([1,2,4,8])
  kjSet([7, 5])
\end{verbatim}
Differences of graphs and dictionaries are allowed, where
{\tt X-Y} produces an object of the same type as {\tt X},
but mixed differences are not allowed when one of the arguments
is a set (yet).
\item{Composition}
with notation {\tt G1*G2} produces the graph containing
{\tt ($s1$,$d2$)} whenever there is an arc
{\tt ($s1$,$d1$)} in {\tt G1} and an arc
{\tt ($d1$,$d2$)} in {\tt G2}.  For example:
\begin{verbatim}
  >>> G1 = kjGraph([(0, 1), (1, 2), (3, 0), (3, 4), (2, 3)])
  >>> G2 = kjGraph([(4, 0), (0, 1), (1, 2), (3, 1), (2, 0)])
  >>> G1*G2
  kjGraph([(0, 2), (1, 0), (3, 1), (3, 0), (2, 1)])
\end{verbatim}
Any two tables can be composed, producing an object of the
more general type.  Composing two sets is a slower way to
compute their intersection.
\item{Transposition}
with notation \verb+~G+ produces the graph containing
{\tt ($d$, $s$)} if and only if G contains {\tt ($s$, $d$)}.
for example
\begin{verbatim}
  >>> G = kjGraph([(0, 0), (3, 2), (6, 4), (20, 1), (23, 3), (26, 5)])
  >>> ~G
  kjGraph([(0, 0), (4, 6), (1, 20), (3, 23), (2, 3), (5, 26)])
\end{verbatim}
Transposition is defined for dictionaries, but 
if there are key collisions the winning pair will be decided
arbitrarily and the resulting table will be dirty.
For example,
\begin{verbatim}
  >>> ~kjDict([("hello","hi"), ("hola","hi"), ("beat it","bye")])
  kjDict([('bye', 'beat it'), ('hi', 'hola')])
\end{verbatim}
This operation is not defined for sets.
\item{Intersection}
produces the set intersection of two structures 
invoked using the notation {\tt X\&Y}.  For example:
\begin{verbatim}
  >>> G = kjGraph([(0,0), (3,2), (6,4), (20,1), (23,3), (26,5), (2,23)])
  >>> G & ~G.tclosure()
  kjGraph([(0, 0), (3, 2), (23, 3), (2, 23)])
\end{verbatim}
Mixed mode intersections between graphs and dictionaries are allowed
{\em producing the less general dictionary type}.  Mixed mode
intersections where one of the arguments is a set is not permitted.
\end{description}

{\bf Note:}
The graph and dictionary operations of composition, reachability,
transitive closure,
and transposition assume that ``right members'' (values) are
hashable.  If any right member is not hashable these functions may
raise a {\tt TypeError}, for example
\begin{verbatim}
  >>> X = kjGraph([ (1,{}) ])
  >>> ~X
  TypeError: unhashable type
\end{verbatim}
Here the empty Python dictionary is not a hashable type, so it could
not be used in the transposed graph as a left member.

\section{On performance}

These structures use a hash table based representation which should
deliver expected good performance for many applications.
Nevertheless, as with all hash implementations there is a
theoretical possibility of very bad worst case performance.
Furthermore, inserts and deletes occasionally cause the
internal structure to resize, so although the average speed for
inserts and deletes is expected to be ``near constant'', once
in a while an insert or delete may be slow.

In addition, since the kjGraph implementation hashes using the
left member only from each graph arc, graphs where many nodes have
a very large number of neighbors may have poor access times.  In this
case it may appropriate to use a ``set of pairs'' 
or a ``dict of sets'' representation
in place of a kjGraph, if this is possible, or some alternative
implementation.

The implementation of {\tt G.tclosure} is 
``quick and dirty (keep it simple, stupid)''
and leaves much room for speed improvements.  It may be
slow for large and complex graphs.  If this is a problem
I might be enticed to improve it, let me know.

Someday I'd like to make the deletion operations faster
(by a constant factor),
but I'm not highly motivated here since I personally tend
to build up tables without ever deleting anything.

\section{Miscellaneous comments}

Once again I'd like to commend Guido and the other Python
contributors on their work.  It's a delight to know that
Python is nice both at the front end and at the back end.

The package is written in C but
descends from an ancestor (not suitable for public
viewing) which was written exclusively in Python.  I wrote
this module (1) as an experimented in extending Python using
C and (2) as an experiment in migrating a Python implementation
to a C implementation.  The result is a package which
I hope may be useful to someone.  

This release is about twice as fast as previous releases thanks to
permiscuous use of C macros in the implementation.  Additionally,
mixed-type operations, coercions, and a few additional methods have
been added in this release.

There is one defined constant in the C code you might
want to play with: {\tt GSIZE} -- the number of
elements of the table heaped together in one ``lump''
(i.e, the size of an unordered subarray of the table).
Roughly speaking if
{\tt GSIZE} is large the table will resize less often, and usually
use space more efficiently.  Generally speaking
larger values will also make the accesses slower, but with
a value less than around 64 this may not always be true on some
machines with fancy memory caching (just guessing here, really).
The default value is 6, which works pretty well on my machines.
{\tt GSIZE} also represents the basic size allocated for the
smallest possible table, so if you expect to use lots of small
sets a large GSIZE may not be advisable.
Don't fiddle with the other constants unless you are willing
to debug possible problems that may result.

\section{Bugs}

Release 2 had a hole in the initializers that caused undefined
behavior.  It has been fixed in 2.1.

Release 2.1 wouldn't link under Python 1.2.  This has been
fixed in 2.2.

The first release would crash on certain graph operations
(transpose, reachability, composition, transitive closure)
applied to graphs that contained unhashable nodes.  Now they raise
an error instead.  Previous releases also had a serious bug
that sometimes corrupted the internal structure of kjSets.
I don't know of any remaining ``real'' bugs
-- the rest of this section discusses possibly confusing
``features.''

As mentioned above in several places, structures that have
been hashed may not be subsequently modified -- attempts to modify
hashed structures will raise {\tt TypeError}.

Mixed mode differences and intersections are not allowed when
one of the arguments is a set (as mentioned).

Some unions and transposes on dictionaries will produce a
dirty dictionary if there are key collisions, and the key
collisions will be decided arbitrarily.  Similarly,
coercing a graph to a dictionary will produce a dirty dictionary
if there are key collisions.  See the section on Dirtiness
above.

The {\tt kjGraph} implementation does not represent nodes
with no edges.  Programmers may work around this either by
wrapping the graph in a class with a node set, or by 
adopting some appropriate
convention that I leave to their infinitely creative imaginations.

Please let me know if you find some other bug or confusing
feature.  At this point
I consider the package to be reasonably well tested, but I
offer {\sc no warrantees}.

\end{document}
